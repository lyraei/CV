
%====================
% EXPERIENCE A
%====================
\subsection{{Konstruktor Elektronik \hfill Maj 2025 --- Obecnie}}
\subtext{AP-FLYER \hfill Warszawa}

\begin{zitemize}
    \item Projektowałem układy cyfrowo-analogowe bezzałogowwych statków powietrznych dla organizacji rządowych i wojska.
    \item Tworzyłem oprogramowanie w języku C dla mikrokontrolerów STM32 zarządzającymi awioniką, zasilaniem i samonaprowadzaniem statków powietrznych.
    \item Montowałem i testowałem zaprojektowane komponenty elektroniczne, zapewniając ich zgodność z ścisłymi wymaganiami technicznymi.
    \item Pracowałem z zespołem interdyscyplinarnym, w tym inżynierami mechanikami i programistami, aby zapewnić integrację systemów elektronicznych z platformą lotniczą.
\end{zitemize}

%====================
% EXPERIENCE B
%====================s
\subsection{{Konstruktor Elektronik \hfill Listopad 2024 --- Maj 2025}}
\subtext{Czaki Thermo-Product \hfill Warszawa}

\begin{zitemize}
    \item Zaprojektowałem precyzyjny przetwornik wilgotności i temperatury oparty na polimerowych pojemnościowych czujnikach wilgotności, osiągając rozdzielczość poniżej 3 fF.
    \item Oprogramowałem mikrokontroler MSP430 do przetwarzania danych i kalibracji czujników, implementując komunikację poprzez MODBUS oraz pętlę prądową 4-20mA, umożliwiając integrację ze standardowymi systemami sterowania.
\end{zitemize}

%====================
% EXPERIENCE C
%====================
\subsection{{R\&D Electronics Engineer \hfill Lipiec 2023 --- Październik 2024}}
\subtext{SVANTEK \hfill Warszawa}

\begin{zitemize}
    \item Zaprojektowałem architekturę zautomatyzowanego testera modułów opartego na Raspberry Pi 5 i układzie FPGA (Xilinx Artix-7), zintegrowanego z dedykowanym sprzętem opracowanym wewnętrznie, co znacząco skróciło czas uruchamiania modułów.
    \item Opracowałem oprogramowanie sterujące testera w językach C++, Rust i Go, w tym własny sterownik I2C slave dla układu RP1 (w języku C), umożliwiający emulację pamięci EEPROM działającej w trybie slave.
    \item Stworzyłem intuicyjny interfejs użytkownika w języku Python, automatyzujący proces testowania poprzez pobieranie testów z serwera, ich wykonywanie oraz raportowanie wyników, co wyeliminowało potrzebę ręcznego uruchamiania.
\end{zitemize}

%====================
% EXPERIENCE D
%====================

% \subsection{{\texorpdfstring{Specialist in operational support department KYC \hfill 06/2020 --- 09/2022}{Specialist in operational support department KYC 06/2020 --- 09/2022}}}
% \subtext{XTB Brokers S.A. \hfill Warszawa}
%
% \begin{zitemize}
% \item Weryfikowałem dokumentację klientów w ramach zespołu wsparcia operacyjnego KYC zgodnie z regulacjami AML/KYC, zapewniając zgodność z przepisami prawa w różnych jurysdykcjach.
% \item Efektywnie współpracowałem w dynamicznym międzynarodowym środowisku finansowym, ściśle współpracując z wielojęzycznym zespołem w celu zapewnienia wsparcia wysokiej jakości.
% \end{zitemize}

%%% Local Variables:
%%% mode: LaTeX
%%% TeX-master: "../resume"
%%% End:
